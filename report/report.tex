\documentclass[11pt]{article}
\begin{document}
	\title{Software Reengineering Project}
	\author{Mitchel Pyl \& Randy Paredis}
	\date{}
	
	\maketitle
	
	\begin{abstract}
		This document is meant as additional information on the reengineering/refactoring of the \texttt{JFreeChart} project, which was the assignment of the \textsf{Software Reengineering} course of 2019, at the \textsf{University of Antwerp}.
	\end{abstract}
	
	\section{Introduction}
	\texttt{JFreeChart} is a Java library that can be used to add/show professional-looking graphs and charts in your Java applications. This inheritly implies that is it useful in a lot of different contexts and scenarios that require this kind of feature.
	
	The ability for such a library for being flexible and expandable with a vast amount of new features would therefore be an incredible advantage for this.
	
	\subsection{Problem at Hand}
	At this point in time, \texttt{JFreeChart} has a wide range of possible graphs, charts and plots it can generate for any kind of data you'd like. Unfortunately the core of this software; i.e. the drawing of a set of points can be considered \textit{legacy code} and is quite solidly hardcoded within the project.
	
	If a user or a client would, for instance, like to create graphs in which every datapoint has a different, predetermined and userdetermined, symbol associated with it\footnote{As was the assignment.}, this would be impossible with the current state of the code. A general fix for this kind of problems within the software-world is to refactor the code so it becomes more flexible and easier to understand.
	
	\section{Project Management}
	
	\section{Project Analysis and Tool Usage}
	\subsection{Repository Mining with CodeScene}
	\subsection{Repository Visualization with Gource}
	
	\section{Refactoring}
	\subsection{Design Recovery}
	\subsection{Design}
	\subsection{Management}
	\subsection{Refactoring}
	
	\section{Preserved Behaviour}
\end{document}